% Formato del título de capítulos y secciones
\titleformat{\chapter}[block]{\titlerule[0.8pt]\normalfont\sf\huge\bfseries}{\thechapter.}{.5em}{\Huge}[{% \titlerule[0.8pt]
}]
\titlespacing*{\chapter}{0pt}{-19pt}{25pt}
\titleformat{\section}[block]{\normalfont\Large\bfseries}{\thesection.}{.5em}{\Large}



% Formato del código fuente con lstlisting
\lstset{
  basicstyle=\ttfamily,
  breaklines=true,
}

% Márgenes
\geometry{
    a4paper,
    margin=2.75cm
}

% Limite de profundidad del índice
\setcounter{tocdepth}{2}

% Indentación de párrafos
\setlength{\parindent}{1cm}

\renewcommand{\lstlistingname}{Extracto de código}
\renewcommand*{\lstlistlistingname}{Índice de extractos de código}

% Formato de código

\definecolor{codegreen}{rgb}{0,0.6,0}
\definecolor{codegray}{rgb}{0.5,0.5,0.5}
\definecolor{codepurple}{rgb}{0.58,0,0.82}
\definecolor{backcolour}{rgb}{0.95,0.95,0.92}

\lstdefinestyle{mystyle}{
    backgroundcolor=\color{backcolour},
    commentstyle=\color{codegreen},
    keywordstyle=\color{magenta},
    numberstyle=\tiny\color{codegray},
    stringstyle=\color{codepurple},
    basicstyle=\ttfamily\footnotesize,
    breakatwhitespace=true,
    breaklines=true,
    captionpos=b,
    keepspaces=true,
    numbers=left,
    numbersep=5pt,
    showspaces=false,
    showstringspaces=false,
    showtabs=false,
    tabsize=2
}

\lstset{style=mystyle}